\documentclass[10pt]{scrartcl}
\usepackage{fullpage,psfrag,amsmath,amsfonts,verbatim,mathtools}
\usepackage{bookman}
\usepackage[small,bf]{caption}
\usepackage{enumitem}
\usepackage{xfrac}
\usepackage{commath}

\newenvironment{sysmatrix}[1]
 {\left[\begin{array}{@{}#1@{}}}
 {\end{array}\right]}
\newcommand{\ro}[1]{%
  \xrightarrow{\mathmakebox[\rowidth]{#1}}%
}
\newlength{\rowidth}
\AtBeginDocument{\setlength{\rowidth}{3em}}
\DeclareOldFontCommand{\bf}{\normalfont\bfseries}{\mathbf}

\newcommand{\ones}{\mathbf 1}
\newcommand{\reals}{{\mbox{\bf R}}}
\newcommand{\integers}{{\mbox{\bf Z}}}
\newcommand{\symm}{{\mbox{\bf S}}} 

\newcommand{\nullspace}{{\mathcal N}}
\newcommand{\range}{{\mathcal R}}
\newcommand{\Rank}{\mathop{\bf Rank}}
\newcommand{\Tr}{\mathop{\bf Tr}}
\newcommand{\diag}{\mathop{\bf diag}}
\newcommand{\card}{\mathop{\bf card}}
\newcommand{\proj}{\mathop{\mathbf Proj}}
\newcommand{\rank}{\mathop{\bf rank}}
\newcommand{\conv}{\mathop{\bf conv}}
\newcommand{\zero}{\mathop{\bf 0}}
\newcommand{\prox}{\mathbf{prox}}

\newcommand{\Expect}{\mathop{\bf E{}}}
\newcommand{\Prob}{\mathop{\bf Prob}}
\newcommand{\Co}{{\mathop {\bf Co}}} 
\newcommand{\dist}{\mathop{\bf dist{}}}
\newcommand{\argmin}{\mathop{\rm argmin}}
\newcommand{\argmax}{\mathop{\rm argmax}}
\newcommand{\epi}{\mathop{\bf epi}} 
\newcommand{\Vol}{\mathop{\bf vol}}
\newcommand{\dom}{\mathop{\bf dom}} 
\newcommand{\intr}{\mathop{\bf int}}
\newcommand{\sign}{\mathop{\bf sign}}

\newcommand{\cf}{{\it cf.}}
\newcommand{\eg}{{\it e.g.}}
\newcommand{\ie}{{\it i.e.}}
\newcommand{\etc}{{\it etc.}}
\bibliographystyle{alpha}

\title{Homework 8 for MATH 312}
\author{Ashok M. Rao}

\begin{document}
\maketitle
\pagenumbering{gobble}

\paragraph{6.2.15} A diagonalizable $A^k$ approaches the zero matrix if and only if every eigenvalue satisfies $-1<\lambda_i<1$. The columns of $A_1$ sum to 1 so it is Markov and has at least one eigenvalue $\lambda=1$. $A_2$ is like a Markov chain without full probability, so we can guess that it's bound to disappear and this is seen by the eigenvalues as well: the characteristic function gives $(\lambda-0.6)^2 = 0.09$ and so the eigenvalues have absolute value below one, and $A_{2}^{\infty}\to 0$.

\paragraph{6.2.18} The eigenvalues for this $A$ are the roots of \[\det{(A-\lambda I)}= \lambda^2 - 4\lambda +3\] so $\lambda=1,3$.
By inspection, the eigenvector corresponding to $\lambda=1$ is $x_1 = (1,1)$ and the eigenvector corresponding to $\lambda=3$ is $x_2=(1,-1)$. This gives,
\[A^k = \frac{1}{2}\begin{bmatrix}1&1\\1&-1\end{bmatrix}\begin{bmatrix}1&0\\0&3^k\end{bmatrix}\begin{bmatrix}1&1\\1&-1\end{bmatrix}\]
with $k=1$ giving the diagonalization.

\paragraph{Strang 6.2.34} A is the rotation matrix so we have
\begin{align} 
\det{(A-\lambda I)} &= \lambda^2 - 2\lambda\cos{\theta} + (\cos^2{\theta} + \sin^2{\theta})\\
	&= \lambda^2 - 2\lambda\cos{\theta} + 1\\
	&= (\lambda - \cos{\theta})^2 + \sin^2{\theta}
\end{align}
So completing the square gives two roots: $\lambda = \cos{\theta}\pm i\sin{\theta}$.To find the eigenvectors see that in $(A-\lambda I)$ only the imaginary term is left on the diagonal. For $\lambda_1=\cos{\theta}+i\sin{\theta}$ multiplying the right column by $-i$ eliminates the left. For the conjugate $\lambda_2$, multiplying the same column by $i$ works. Thus we have,
\[A^{n} = \frac{1}{2i}\begin{bmatrix}1&1\\-i& i\end{bmatrix}\begin{bmatrix}\text{cis}(\theta)&0\\0&\text{cis}(-\theta)\end{bmatrix}^n\begin{bmatrix}i&-1\\i&1\end{bmatrix}=\begin{bmatrix}\cos{n\theta}& -\sin{n\theta}\\ \sin{n\theta}& \cos{n\theta}\end{bmatrix}\]
This uses the fact that $\cos{\theta}$ is even and $\sin{\theta}$ is odd so $\text{cis}(-\theta)$ returns the conjugate.  

\paragraph{Eigenvectors of ``nearly symmetric" matrices} The given matrix has the characteristic polynomial
\[\det{(A-\lambda I)} = \lambda^2 -(2+\epsilon)\lambda + (\epsilon+1) \implies (\lambda-1)^2 = (\lambda-1)\epsilon\]
so we have $\lambda=1,\epsilon+1$ as the two eigenvalues. These correspondingly give the eigenvectors $(1,0)$ and $(1,1)$. The angle between these vectors is $\cos^{-1}\left(\frac{1}{\sqrt{2}}\right)=45\text{ degrees}$, which is not orthogonal. 

\paragraph{6.4.13}
Using the characteristic equations for $S$ and $B$,
\[\det{(S - \lambda I)}=(\lambda-4)(\lambda-2),\quad \det{(B-\lambda I)}=\lambda(\lambda-25)\]
it is clear that $S$ has eigenvectors $(1,1)$ and $(-1,1)$ for $\lambda=4,2$ respectively and similarly that $B$ has eigenvectors $(3,4)$ and $(-4,3)$ for $\lambda=25,0$. Thus,
\[S = (4/2)(1,1)(1,1)^T+(2/2)(-1,1)(-1,1)^T,\quad
B = (25/25)(3,4)(3,4)^T\]
The fractional coefficients are left unsimplified to highlight the distinction between the eigenvalue numerators and normalizing denominators (the eigenvector corresponding with the zero eigenvalue for $B$ is removed). This gives the $Q\Lambda Q^{T}$ form of $S$ and $B$ since the eigenvectors are orthogonal and may be added as shown to recompose into the original matrix. Since $Q$ is symmetric we use that $QQ^T = I$. 


\end{document}