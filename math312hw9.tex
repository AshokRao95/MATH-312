\documentclass[10pt]{scrartcl}
\usepackage{fullpage,psfrag,amsmath,amsfonts,verbatim,mathtools}
\usepackage{bookman}
\usepackage[small,bf]{caption}
\usepackage{enumitem}
\usepackage{xfrac}
\usepackage{commath}

\newenvironment{sysmatrix}[1]
 {\left[\begin{array}{@{}#1@{}}}
 {\end{array}\right]}
\newcommand{\ro}[1]{%
  \xrightarrow{\mathmakebox[\rowidth]{#1}}%
}
\newlength{\rowidth}
\AtBeginDocument{\setlength{\rowidth}{3em}}
\DeclareOldFontCommand{\bf}{\normalfont\bfseries}{\mathbf}

\newcommand{\ones}{\mathbf 1}
\newcommand{\reals}{{\mbox{\bf R}}}
\newcommand{\integers}{{\mbox{\bf Z}}}
\newcommand{\symm}{{\mbox{\bf S}}} 

\newcommand{\nullspace}{{\mathcal N}}
\newcommand{\range}{{\mathcal R}}
\newcommand{\Rank}{\mathop{\bf Rank}}
\newcommand{\Tr}{\mathop{\bf Tr}}
\newcommand{\diag}{\mathop{\bf diag}}
\newcommand{\card}{\mathop{\bf card}}
\newcommand{\proj}{\mathop{\mathbf Proj}}
\newcommand{\rank}{\mathop{\bf rank}}
\newcommand{\conv}{\mathop{\bf conv}}
\newcommand{\zero}{\mathop{\bf 0}}
\newcommand{\prox}{\mathbf{prox}}

\newcommand{\Expect}{\mathop{\bf E{}}}
\newcommand{\Prob}{\mathop{\bf Prob}}
\newcommand{\Co}{{\mathop {\bf Co}}} 
\newcommand{\dist}{\mathop{\bf dist{}}}
\newcommand{\argmin}{\mathop{\rm argmin}}
\newcommand{\argmax}{\mathop{\rm argmax}}
\newcommand{\epi}{\mathop{\bf epi}} 
\newcommand{\Vol}{\mathop{\bf vol}}
\newcommand{\dom}{\mathop{\bf dom}} 
\newcommand{\intr}{\mathop{\bf int}}
\newcommand{\sign}{\mathop{\bf sign}}

\newcommand{\cf}{{\it cf.}}
\newcommand{\eg}{{\it e.g.}}
\newcommand{\ie}{{\it i.e.}}
\newcommand{\etc}{{\it etc.}}
\bibliographystyle{alpha}

\title{Homework 9 for MATH 312}
\author{Ashok M. Rao}

\begin{document}
\maketitle
\pagenumbering{gobble}

\paragraph{Strang 6.5.22} For $S_1$ the characteristic function is $p(\lambda) = \lambda^2 - 10\lambda + 9 = (\lambda-9)(\lambda-1)$ giving $\lambda=1,9$. For $S_2$ we have $p(\lambda)=\lambda^2 - 20\lambda + 64 = (\lambda-4)(\lambda-16)$ so $\lambda=4,16$. By diagonalizing, the square root falls on the eigenvalues giving for $S_1$
\[A_1^{1/2} = \frac{1}{\sqrt{2}}\begin{bmatrix}1&1\\1&-1\end{bmatrix}
				\begin{bmatrix}3&0\\0&1\end{bmatrix}
				\begin{bmatrix}1&1\\1&-1\end{bmatrix}\frac{1}{\sqrt{2}}=\begin{bmatrix}2&1\\1&2\end{bmatrix}\]
And for $S_2$,
\[A_2^{1/2} = \frac{1}{\sqrt{2}}\begin{bmatrix}1&1\\1&-1\end{bmatrix}
				\begin{bmatrix}4&0\\0&2\end{bmatrix}
				\begin{bmatrix}1&1\\1&-1\end{bmatrix}\frac{1}{\sqrt{2}}=\begin{bmatrix}3&1\\1&3\end{bmatrix}\]
The roots normalize $QQ^T$ and it is clear that in both cases $S=A^T A$, in fact $S=A^2$ since $A=A^T$ in this case.

\paragraph{Strang 6.5.35} For positive definite $S$ with eigenvalues indexed in descending order, the matrix $(\lambda_1 I - S)$ will have eigenvalues $\lambda_1 - \lambda_i \geq 0$ so the matrix is positive semidefinite. (These are the eigenvalues because the characteristic function $p(\lambda+\lambda_1)$ is just shifted across).

\paragraph{Strang 7.1.6}
$A$ has eigenvalues $\lambda=0,4$ either by a trivial characteristic function or that these are the only values so that $\det{A}=0$ while $\Tr{A}=4$. Very similarly, $A^T A$ has eigenvalues 0 and 25, giving $\sigma_1=5$ and $\sigma_2=0$. corresponding to eigenvectors   $v_1=(2,1)/\sqrt{5}$ and $v_2 = (1,-2)/\sqrt{5}$. These are orthogonal (orthonormal, now) as they must be since $A^T A$ is symmetric. $AA^T$ has the same eigenvalues as $A^T A$ but with eigenvectors $u_1=(1,2)/\sqrt{5}$ and $u_2=(2,-1)/\sqrt{5}$.

\paragraph{SVD example} To find the SVD we first calculate $A^T A$ and $A A^T$ for the singular values and orthogonal bases,
\[A^T A = \begin{bmatrix}3& 2\\2& 3\\2&-2\end{bmatrix}\begin{bmatrix}3&2&2\\2&3&-2\end{bmatrix} = \begin{bmatrix}13&12&2\\12&13&-2\\2&-2&8\end{bmatrix},\quad A A^T = \begin{bmatrix}17&8\\8&17\end{bmatrix}\]
It seems easier to find eigenvalues from $A A^T$, which are shared with $A^T A$ (except for $\lambda=0$ in the nullspace). This gives $det{(AA^T - \lambda I)} = (\lambda-9)(\lambda-25)$ and so we have $\sigma_1, \sigma_2 = 5,3$.  These correspond to eigenvectors for $A^T A$ which are respectively $v_1=(1,1,0)$ and $v_2=(1,-1,4)$. Likewise for $A A^T$ we have $u_1 = (1,1)$ and $u_2 = (-1, 1)$. By the fact that $Av_i = \sigma_i u_i$ and normalizing, we can get the SVD
\[A= \frac{5}{\sqrt{2}}(1,1)\frac{1}{\sqrt{2}}(1,1,0)^T + \frac{3}{\sqrt{2}}(-1,1)\frac{1}{\sqrt{18}}(1,-1,4)^T = \frac{5}{2}\begin{bmatrix}1&1&0\\1&1&0\end{bmatrix}+\frac{1}{2}\begin{bmatrix}-1&1&-4\\1&-1&4\end{bmatrix}\]
Or put in the form $A=U\Sigma V^T$,
\[U\Sigma\ V^T = \frac{1}{\sqrt{2}}\begin{bmatrix}1&-1\\1&1\end{bmatrix}\begin{bmatrix}5&0&0\\0&3&0\end{bmatrix}\frac{1}{3\sqrt{2}}\begin{bmatrix}3&3&0\\-1&1&-4\\-2\sqrt{2}&2\sqrt{2}&\sqrt{2}\end{bmatrix}\]

\paragraph{Strang 7.4.11} For $A=(3,4,0)^T$,
\[A^T A = \begin{bmatrix}9&12&0\\12&16&0\\0&0&0\end{bmatrix},\quad A A^T = [25],\quad U\Sigma V^T = \begin{bmatrix}1\end{bmatrix}\begin{bmatrix}5&0&0\end{bmatrix}\frac{1}{5}\begin{bmatrix}3&4&0\\0&0&1\\-4&3&0\end{bmatrix}\]
Singular value $\sigma_1=5$ corresponds to the vector $v_1 =(1/5)(3,4)$ and $\sigma_2=0$ corresponds to $v_2=(1/5)(4,-3)$. The pseudoinverse can now be calculated like $A^{+} = V\Sigma^{+}U^T$,
\[ A^{+}= \frac{1}{5}\begin{bmatrix}3&0&-4\\4&0&3\\0&1&0\end{bmatrix}\begin{bmatrix}1/5\\0\\0\end{bmatrix}=\begin{bmatrix}3/25\\4/25\\0\end{bmatrix},\quad
A^{+} A = \begin{bmatrix}9/25&12/25&0\\12/25&16/25&0\\0&0&0\end{bmatrix},\quad A A^{+} = \begin{bmatrix}1\end{bmatrix}\]
\end{document}