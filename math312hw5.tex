\documentclass[10pt]{article}
\usepackage{fullpage,graphicx,psfrag,amsmath,amsfonts,verbatim,mathtools}
\usepackage[small,bf]{caption}
\usepackage{enumitem}

\newenvironment{sysmatrix}[1]
 {\left[\begin{array}{@{}#1@{}}}
 {\end{array}\right]}
\newcommand{\ro}[1]{%
  \xrightarrow{\mathmakebox[\rowidth]{#1}}%
}
\newlength{\rowidth}
\AtBeginDocument{\setlength{\rowidth}{3em}}

\newcommand{\ones}{\mathbf 1}
\newcommand{\reals}{{\mbox{\bf R}}}
\newcommand{\integers}{{\mbox{\bf Z}}}
\newcommand{\symm}{{\mbox{\bf S}}} 

\newcommand{\nullspace}{{\mathcal N}}
\newcommand{\range}{{\mathcal R}}
\newcommand{\Rank}{\mathop{\bf Rank}}
\newcommand{\Tr}{\mathop{\bf Tr}}
\newcommand{\diag}{\mathop{\bf diag}}
\newcommand{\card}{\mathop{\bf card}}
\newcommand{\rank}{\mathop{\bf rank}}
\newcommand{\conv}{\mathop{\bf conv}}
\newcommand{\zero}{\mathop{\bf 0}}
\newcommand{\prox}{\mathbf{prox}}

\newcommand{\Expect}{\mathop{\bf E{}}}
\newcommand{\Prob}{\mathop{\bf Prob}}
\newcommand{\Co}{{\mathop {\bf Co}}} 
\newcommand{\dist}{\mathop{\bf dist{}}}
\newcommand{\argmin}{\mathop{\rm argmin}}
\newcommand{\argmax}{\mathop{\rm argmax}}
\newcommand{\epi}{\mathop{\bf epi}} 
\newcommand{\Vol}{\mathop{\bf vol}}
\newcommand{\dom}{\mathop{\bf dom}} 
\newcommand{\intr}{\mathop{\bf int}}
\newcommand{\sign}{\mathop{\bf sign}}

\newcommand{\cf}{{\it cf.}}
\newcommand{\eg}{{\it e.g.}}
\newcommand{\ie}{{\it i.e.}}
\newcommand{\etc}{{\it etc.}}
\bibliographystyle{alpha}

\title{Homework 5 for MATH 312}
\author{Ashok M. Rao}

\begin{document}
\maketitle
\pagenumbering{gobble}

\paragraph{(Strang 3.3.4)}
To find the general solution we first reduce to $Rx=d$ to find the particular solution:
\[
\begin{sysmatrix}{rrrr|r}
1& 3& 1& 2& 1\\
2& 6& 4& 8& 3\\
0& 0& 2& 4& 1
\end{sysmatrix}\rightarrow
\begin{sysmatrix}{rrrr|r}
1& 3& 1& 2& 1\\
0& 0& 1& 2& 1/2\\
0& 0& 0& 0& 0
\end{sysmatrix}\rightarrow
\begin{sysmatrix}{rrrr|r}
1& 3& 0& 0& 1/2\\
0& 0& 1& 2& 1/2\\
0& 0& 0& 0& 0
\end{sysmatrix}
\]
Here $x_1, x_3$ are pivots with $x_2, x_4$ free.  Solving for $d$ the particular solution is $x_p = (1/2, 0, 1/2, 0)$. $N(R)$ is spanned by vectors $(-3, 1, 0, 0)$ and $(0, 0, -2, 1)$, found by setting $x_2$ or $x_4$ to 0. The general solution is thus $x = (1/2, 0, 1/2, 0) + c_1(-3,1,0,0) + c_2(0,0,-2,1)$.

\paragraph{Rank of a matrix}
Reduce $A$ by subtracting $A_1$ from $A_2$ and $A_3$ and $A_2$ from $A_3$ resulting in R (below). If $q=2$ then the final row is redundant and $r=2$; the nullspace solution formed on the line $c(1,1,-1)$. Otherwise the matrix is full rank with $n=m=r=3$ with one solution via substitution.
\[R =\begin{sysmatrix}{rrr|r}
	1& 0& 1& b_1\\
	0& 1& 1& b_2-b_1\\
	0& 0& q-2& b_3-b_2
\end{sysmatrix}\]

\paragraph{(Strang 3.3.24)}
The matrices $A,B,C,D$ below solve the respective conditions. For $A$, $b$ has a solution if and only if $b_1=b_2$. $B$ admits infinitely many solutions on the line $x_1 + x_2 = b$. $C$ admits any $x\in\reals^n$ if $b=0$ and nothing otherwise. $D$ is always solved by $(b_1, b_2/2)$.
\[Ax=\begin{bmatrix}a& a\end{bmatrix}^T\begin{bmatrix}x\end{bmatrix},\quad
Bx= \begin{bmatrix}1& 1\end{bmatrix}\begin{bmatrix}x_1&x_2\end{bmatrix}^T,\quad
Cx=\zero x,\quad 
Dx= \begin{bmatrix}e_{1}^{T}& 2e_{2}^{T}\end{bmatrix}\begin{bmatrix}x_1 x_2\end{bmatrix}^{T}
\]


\paragraph{(Strang 3.3.30)} The particular solution follows from the Gauss-Jordan form on the right. By substitution and a zero free variable, $x_p = (-4,3,0,2)$ is obtained as a solution. The homogenous solutions are obtained in the nullspace with $x_3=1$ yielding $x_n = c(-2,0,1,2)$ for $x = (-4,3,0,2)+c(-2,0,1,2)$. 
\[
\begin{sysmatrix}{rrrr|r}
1& 0& 2& 3& 2\\
1& 3& 2& 0& 5\\
2& 0& 4& 9& 10
\end{sysmatrix}\rightarrow
\begin{sysmatrix}{rrrr|r}
1& 0& 2& 3& 2\\
0& 3& 0& -3& 3\\
0& 0& 0& 3& 6
\end{sysmatrix}\rightarrow
\begin{sysmatrix}{rrrr|r}
1& 0& 2& 0& -4\\
0& 3& 0& 0& 9\\
0& 0& 0& 3& 6
\end{sysmatrix}\rightarrow
\begin{sysmatrix}{rrrr|r}
1& 0& 2& 0& -4\\
0& 1& 0& 0& 3\\
0& 0& 0& 1& 2
\end{sysmatrix}
\]


\paragraph{(Strang 3.4.13)}
The row spaces of $A$ and $U$ are equivalent and 2 dimensional, since they are each spanned by $(1,1,0)$ and $(0,2,1)$. The respective column spaces share one basis vector, $(1,2,0)$, and also span a 2 dimensional space since $a_{1}^{T} + 2a_{3}^{T} = a_{2}^{T}$ and $u_{1}^{T} + u_{3}^{T} = u_{2}^{T}$. 

\paragraph{(Strang 3.4.33)}
The functions $y_1 = \exp\{2x\}$ and $y_2 = x$ span the solution spaces for $y'/y=2$ and $y'/y=1/x$ respectively. These  equations have unit dimensional solution spaces as they are first order.

\paragraph{(Strang 3.4.34)}
$y_1(x),y_2(x)$, and $y_3(x)$ may be polynomials. If they each have degree one, like $y_i(x)=c_i x$, their span has dimension 1.  If otherwise any two have degree 1 and the other degree 2 (like $ax$, $bx$, and $cx^2$) then the span has dimension 2. If all three have unique degrees 1,2, and 3 (like $x$, $x^2$, $x^3$) then the span has dimension 3.

\end{document}